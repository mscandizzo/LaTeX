% structure.tex - A simple article to illustrate document structure
\documentclass{article}
\usepackage[utf8]{inputenc}
\usepackage[english]{babel}
\usepackage{ragged2e}

\usepackage{blindtext}

\begin{document}
\title{ \LaTeX{} for Lambda Students}
\author{Mariano Scandizzo,\\
	Scandizzo \& Partners, LLC\\
	Portland, Maine\\
	\texttt{mariano.scandizzo@spartners.me}}
\date{\today}
\maketitle
\section{Introduction}
\justifying
\LaTeX{} (sometimes written `LaTeX' and always pronounced `lay tech')
 is a document markup language. Besides being easy on the eyes, 
 \LaTeX{} is the industry standard for modern technical documents in 
 mathematics, computer science, machine learning, and data science.
 At Lambda we believe strongly in learn-ing by doing, so we'll just 
 right into the assignment. First, go to overleaf.com and make an account.
 Create a new, blank project and delete any text that automatically comes
 in the document. Then visit the excellent site Getting to grips with \LaTeX{}
 by Andrew Roberts and work through sections 1, 2, 9, and 10 inside that blank
 project. Now make another document with a header like the one on this page
 (use your own name and replace my organization's name with Lambda School).
 Inside that document, use everything you've learned to typeset the BAC-CAB
 rule from vector algebra
\begin{equation}
	\overrightarrow{A} x \left(\overrightarrow{B} x \overrightarrow{C} \right) = \overrightarrow{B} x \left(\overrightarrow{A} x \overrightarrow{C} \right) + \overrightarrow{C} x \left(\overrightarrow{A} x \overrightarrow{B} \right)
\end{equation}
and the following illustration of the chain rule:
 \begin{eqnarray*}
       \frac{\partial \sin \left(x^2 + xy \right)}{\partial x} & = & {\partial \frac{\sin \left(x^2 + xy \right)}{\left(x^2 + xy \right)}} {\partial \frac{\left(x^2 + xy \right)}{\left(x \right)}} \\
       & = & \cos \left(x^2 + xy \right) \left(\partial \frac{x^2}{x} + \partial \frac{x}{y} y \right) \\
       & = & \cos \left(x^2 + xy \right) \left(2x + y \right) 
 \end{eqnarray*}
Pay attention to the small details:
\begin{itemize}
	\item Parenthesis should be large enough for the objects inside of them (this can be automated!).
	\item Write sin, not sin. Write cos, not cos. In derivatives, write d, not d.
\end{itemize}


\end{document}